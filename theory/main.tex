\documentclass[letter,10pt]{article}

\setcounter{tocdepth}{4} 
\setcounter{secnumdepth}{4}

\usepackage[top=1.0in, bottom=1.0in, left=1.0in, right=1.0in]{geometry}
%\usepackage[utf8]{inputenc} 
%\usepackage{amssymb}
%\usepackage{caption} 
%\usepackage{enumerate}
%\usepackage{lineno}
%\usepackage{mathtools}
%\usepackage{tikz}

\usepackage{amsmath} 
\usepackage{amssymb} 
\usepackage{array}
\usepackage{enumitem}
\usepackage{graphicx}
\usepackage{natbib} 
\usepackage{subcaption} 
\usepackage{booktabs} 
\usepackage{authblk}


%\linenumbers
%\linespread{1.5}



\pdfminorversion=4

\makeatother

\title{An Adaptation of SE Min-Cut for Hyperspectral Image Segmentation}

\author[1]{G D Portwood}
\author[2]{A Nichols}
\affil[1]{Department of Mechanical and Industrial Engineering}
\affil[2]{Department of Electrical and Computer Engineering}

\begin{document}

\maketitle

\begin{abstract}

\end{abstract}

\section{Introduction}

\section{Algorithm}
\citet{estrada04} have defined an affinity measure $a_{i,j}$ between all points
$\vec{x}_i$ and $\vec{x}_j$ where $||\vec{x}_i-\vec{x}_j||_1=1$ such that
\begin{equation}
  a_{i,j}=e^{\frac{-(I^\prime(\vec{x}_i)-I^\prime(\vec{x}_j))^2}{2 \sigma^2}},
\end{equation}
where $I^\prime : \mathbb{R}^1 \mapsto \mathbb{R}^1$ is the scalar intensity and
$\sigma$ is a constant which normalizes variation between pixels.  As an
extension to a spectral domain, we have accomodated for a dimension of size $k$
which corresponds to the number of bands in the image, with 
\begin{equation}
  a_{i,j}=e^{\frac{-||I^\prime(\vec{x}_i)-I^\prime(\vec{x}_j)||_2^2}{2 \sigma^2}},
\end{equation}
where $I^\prime : \mathbb{R}^1 \mapsto \mathbb{R}^1$ and $|| \cdot ||_2$ denotes
the Euclidean norm.






\section{Application}

\section{Conclusion}

\bibliographystyle{apalike}
\bibliography{bib}

\end{document}
